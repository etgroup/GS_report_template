\section{前言}

电容测量可分为两种情况:浮地电容和接地电容的测量。浮地电容是指电容两端不具备特定电势,测量相对容易,有多种可选的测量电路,都具备抗杂散电容的能力。如ECT中的电容测量就是典型的浮地电容测量。各电极可根据测量需要施加特定电势或进行接地处理。接地电容指电容的一端接地。由于杂散电容多是对地电容,其直接与接地电容并联;如果无法有效区别被测接地电容与杂散电容,那么抗杂散电容能力也就无从谈起。对地电容的测量无法使用一般的电容测量仪器进行直接测量。如LCR表、阻抗分析仪都需要将其阻抗两端接入其测量端口,施加特定的电势。接地电容,或受条件限制,或受传感器原理限制,无法提供浮置选项。

浮地电容的检测已经有很多成熟的测量电路,如直流充放电电路【@季寒 S. M. Huang,Charge/discharge ... 】、交流充放电电路【】【@季寒 W. Q. Yang, New AC based...】、谐振电路【@季寒 】等,且都具备抗杂散电容的能力。

在一些应用中,如针对导电溶液的电容式液位计、液膜厚度测量仪等,电容器的一端实际上就是导电溶液。由于导电溶液多通过金属容器壁与大地相连,致使这类传感器只能做为接地电容。为此,发展出一些针对接地电容的测量方法与电路,如【@季寒 检索英文文献】

\section{方法}
\subsection{基本原理}

【@季寒 结合原理框图,说明方法。】

此处,原理框图只包含最少的要素:1、非理想电流源(由理想电流源与输出阻抗构成);2、被测接地电容Cx;3、与Cx并联的杂散电容Cp。
图1 基于电流源的接地电容测量方法

【@季寒 给出公式,电路输出值与被测电容之间的关系。】


\subsection{电流源}
【采用电路原理图和公式描述可用的VCCS电路】

Howland 型
改进的Howland型
CCII型
【A wide-band AC-coupled current source for electrical impedance tomography】

\subsection{电缆屏蔽}
引用图1,进一步说明电缆屏蔽的必要性。


\section{结果}
\subsection{电路仿真结果}
电路仿真软件给出仿真数据及信号图。

\subsection{实验结果}
结合实验给出测量结果。这里,我们可以:
用液位计构建一个液位与电容成正比的电容器,作为研究目标;
使用同轴电缆(100 pF/m)作为被测电容。
注意:常用的贴片电容,精度为10~20%,不可作为比对标准。



\section{讨论}

\section{结论}







参考文献
@季寒 以下及文中提到的文献,保存到Jabref,备用。

A novel interface circuit for grounded capacitive sensors with feedforward-based active shielding

A simple analog front-end circuit for grounded capacitive sensors with offset capacitance

Liquid-level measurement system based on a remote grounded capacitive sensor

Comparison of three current sources for single-electrode capacitance measurement

