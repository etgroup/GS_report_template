\section{研究生培养}

\subsection{周报制度}

为督促研究生抓紧时间,要求每周撰写进度报告,于每周六晚E-mail给老师;由导师根据情况,对研究生一周的工作情况、成果进行汇总、修改和点评,并群发周知。



\section{每周进度报告模板}


一、下周工作重点明细及时间安排

【结合论文推进过程中的各个Deadline时间节点,明确列写各项工作明细,各项耗时也可用半天为单元进行计算;任务繁杂时需进行细分,细分后的单项工作耗时不超过2天】

(1)完成DECT论文Introduction的思路确定及写作,含主要参考文献;(预计耗时2天,周一、周二完成)

(2)完成摘要及结论的撰写;

(3)查阅文献确定实验条件,补做相关实验,完成实验数据的处理及实验分析部分文字的撰写;(2、3项合计耗时2天,周三、周四完成)

(4)根据老师反馈意见及同学互查意见修改全文;对论文内容进行比对查重,根据查重结果修改论文;(预计耗时1天,周五完成)

(5)根据目标期刊更新参考文献,完成在线投稿,生成proof。(预计耗时1天,周六完成)

二、本周整体进度及内容明细

(1)完成离散IEEE Sensors J论文的查重、终稿修改以及投稿;

(2)完成MST论文理论推导部分的图形、实验部分的文字;

(3)查阅文献确定接地电容测量方法相关内容的可行性;

(4)协助张威扬完成稿件中的文字修改及补做实验。


附:上周所列写的“下周工作重点明细及时间安排”

【将上周邮件中的对应内容拷贝到此处】


注:石墨平台工作状态、工作日志及链接文档的更新时间为周二、周四及周六。

\section{工作进度管理平台}
\subsection{石墨平台的使用}

研究生工作进度Excel表格(以下称进度表)使用“石墨平台/APP”存放、管理。
每位研究生都有专属的进度表格,需要花一定的时间熟悉其使用方法。

\begin{enumerate}
\item 研究生须在每周二、四、六更新工作进度及工作文档;导师每周日做一次工作进度汇总,邮件群发给所有人。
\item 遇到问题时,须先尝试两种以上方法,再将问题及方法列在表格中,与导师讨论;
\item 使用GitHub或天大云盘存放文档,将pdf/word文件的链接填入石墨文档中。要求:在石墨文档中,点击1次即能打开链接的报告文件。
\item 中文文档采用天大学位论文格式(方便日后形成学位论文),英文文档可采用IEEE期刊格式;
\item 石墨文档可以直接在文档中使用类似微信的@某某的功能,可使用该功能提醒我特别注意。
\end{enumerate}


\subsection{GitHub的使用}

GitHub的相关介绍可以网络上查阅。
使用GitHub,要求:

\begin{enumerate}
\item 建立一个GitHub.com的账号,建议使用@tju.edu.cn的邮箱
\item 理解版本管理的概念
\item 会使用GitHub Desktop客户端来管理工作报告、论文草稿、程序代码
\item 掌握基本操作:clone, add, commit, push, pull, merge 
\end{enumerate}

使用GitHub管理LaTeX文档,在网页端打开其生成的pdf文件后,点击“Download”,在浏览器的地址栏就会有这个pdf的下载链接。
可以将这个链接放入进度表中。


\subsection{天大云盘的使用}

目前,天大云盘只对老师开放,有些不便,但是速度快,适合传递较大的文件。具体使用方法如下:

\begin{enumerate}
\item 登录老师提供的链接,将报告文档放入指定的文件夹;
\item 上传pdf文件后,在网页上打开pdf文件,复制浏览器地地址栏的内容,将这个链接放入进度表中。
\end{enumerate}


\section{其他措施}

根据论文完成情况,给予一定的实验室奖学金,以示鼓励。






