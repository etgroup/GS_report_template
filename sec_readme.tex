
\textcolor[rgb]{0.00,0.07,1.00}{注意:研究生平时撰写报告应遵循学术论文的一般写作规范,而不是敷衍导师。这样做的优势是:每一份报告都将成为你毕业论文的章/节。
}
\section{使用报告模板}
\subsection{报告撰写}

在根目录下,包括两个主文件:
\begin{verbatim}
   draft_en.tex:使用IEEE期刊格式,用于撰写英文报告;
   draft_cn.tex:使用天津大学学位论文格式,用于撰写中文报告。
\end{verbatim}

其中都包含以下类似语句:

\begin{verbatim}
  \section{Method}


\subsection{Cylindrical capacitor}
The capacitance of a cylindrical capacitor is described by:

\begin{eqnarray}
C_{cc} = \frac{2\pi h \varepsilon_0}{\ln(b/a)}
\label{eq:cylindrical_capacitor}
\end{eqnarray}
where $\varepsilon_0$ is the dielectric permittivity of air, $h$ is the axial length of active screen, $a$ and $b$ are the radius of measuring electrodes and active screen, respectively.



Therefore, it can be assumed that

\begin{eqnarray}
C_{d1}&=&C_{d2}=\cdots=C_{di}\\
C_{cc}&=&\sum_{i=1}^{n_E} C_{di}
\label{eq:cd}
\end{eqnarray}
where $C_{di}$ is the capacitance between the $i^{th}$ measuring electrode  and the active screen.

In consideration of the gaps between measuring electrodes and radial screens and the edge effect caused by finite $h$, $C_{di}$  can be estimated by:

\begin{eqnarray}
C_{di} = k_c\frac{C_{cc}}{n_E}=\frac{2k\pi h \varepsilon_0}{\ln(b/a)n_E}
\label{eq:cna}
\end{eqnarray}
where $k_c$ is a correction factor related to sensor geometry, $n_E$ is the number of measuring electrodes.

\subsection{Circuit output}


Therefore, the output voltage can be described by:

\begin{eqnarray}
V_o(t) = -\frac{j\omega R_f(C_m - kC_{d2})}{j\omega C_f R_f +1}V_i(t)
\label{eq:voa}
\end{eqnarray}
where $\omega$ is the applied angular frequency, $R_f$ and $C_f$ are the feedback resistor and capacitor.

By introducing $k_0 = {j\omega R_f}/(j\omega C_f R_f +1)$, Eq(\ref{eq:voa}) can be simplified into:

\begin{eqnarray}
V_o(t) =-k_0(C_m - kC_{d2})V_i(t)
\label{eq:vob}
\end{eqnarray}


\subsection{Noise analysis}


In consideration that
\begin{eqnarray}
|j\omega R_{s3}(C_m+C_d+C_{s3}+C_{s4})| &\ll&1,\nonumber\\
|j\omega R_{s4}(C_m+C_d+C_{s3}+C_{s4})| &\ll&1,\nonumber\\
|j\omega R_fC_f| &\gg&1\nonumber
\label{eq:approx_cond}
\end{eqnarray}
the equations for noise gains can be properly simplified.

The total RMS noise can be expressed by

\begin{eqnarray}
E_{t} = \sqrt{\sum_{i=1}^{q} E_{n}^2(i)}[V_{RMS}]
\label{eq:vrms_sum}
\end{eqnarray}
where $E_{n}(i)$ represents the $i^{th}$ RMS noise source, \textit{i.e.} the passive elements and the intrinsic noise sources in the op-amp.

Each noise source generates a noise density at the inverting or non-inverting input of the op-amp, which can be  calculated by:

\begin{eqnarray}
E_n = \sqrt{\int_{0}^{\omega_c}|A_n(\omega)|^2 e_{n}^{2}(\omega)d \omega}[V_{RMS}]
\label{eq:vrms}
\end{eqnarray}
where $A_n(\omega)$ is the noise gain, $\omega_c$ is the cut-off frequency of the following conditioning circuit.

\subsection{Sensitivity distribution}

In ECT, the sensitivity distribution of electrode pair $(i, j)$ $\mathbf{S}_{i,j}(k)$ is defined as:

\begin{eqnarray}
\mathbf{S}_{i,j}(k)=\mu (k)\frac{C_{i,j}(k)-C_{i,j}^l}{\Delta C_{i,j}\Delta \varepsilon}
\label{eq:sensitivity}
\end{eqnarray}
where
\begin{eqnarray}
\Delta C_{i,j} &=&C_{i,j}^h-C_{i,j}^l\\
\Delta \varepsilon &=&\varepsilon^h-\varepsilon^l
\label{eq:sensitivity2}
\end{eqnarray}
and where $C_{i,j}(k)$ is the capacitance when the $k^{th}$ element has the high dielectric constant $\varepsilon^h$ and all the other elements have low dielectric constant $\varepsilon^l$. And $C_{i,j}^h$ and $C_{i,j}^l$ are the capacitances when the pipe is filled with  high and low permittivity materials, respectively. $\Delta C_{i,j}$ is, therefore, the full scale capacitance range.

\end{verbatim}

其功能是将此文件包含入主文件内。
在使用过程中,直接编辑文件\verb+sec_method.tex+中的内容,再利用pdflatex功能编译主文件即可生成:
\begin{verbatim}
   draft_en.pdf:英文报告;
   draft_cn.pdf:中文报告。
\end{verbatim}

可根据需要,将pdf报告提交给导师。

\subsection{修改题目及作者姓名}

在\verb+\preface\cover.tex+文件中,可修改报告题目及作者姓名。

\subsection{参考文献:\LaTeX、BibTeX 与Jabref配合使用}


\LaTeX 、BibTeX与Jabref配合使用,可方便的写出格式非常漂亮的论文、报告,甚至学位论文。

\begin{description}
  \item[\LaTeX] 可Google或百度,常用的编写软件有CTex套装、TexLive、TexStudio等。
  \item[BibTex] 是LaTex的一个组件,用于文献引用。其中,.bib是文献引用信息文件。
  \item[Jabref] 用于管理BibTex的一个软件,跨平台。
\end{description}

\subsection{文献管理}
长期积累下来,参考文献的数量是非常可观的,单靠人的记忆力显然不够。借助文献管理软件,可以很好的解决这个问题。推荐两个:ENDNOTE和Jabref,网上有很多教程。

Jabref和 \LaTeX 配合的很好。

以前的做法是:每篇论文建个文件夹,放tex文件,以及图片、引用的论文 和bib(参考文献列表)等。问题是,同一篇参考文献可能要会在不同的论文中引用, 每次引用Copy一份bib文件,在管理上 很不方便。

给大家推荐一种方法:

\begin{enumerate}
  \item 规划一个写论文专用的文件夹,用于存所有tex文件(以及tex相关的cls文件)、图片(有的图片也可能会重复使用)、参考文献(bib、bst文件,及pdf)等,例如我用c:/local;
  \item 在此文件夹下,可新建ref、fig、bib等子文件夹,分别存放上述文件;
  \item 当开始一篇新文章A时,在其他地方新建一个文件夹:20180101\_A,放上Tex源文件;
  \item 在tex文件中需要插入文献/图片时,把c:/local 同时加进去,见下例。
\end{enumerate}

\begin{verbatim}
\documentclass[twoside]{c:/local/linux_thesis}  %引用cls模板
...
\graphicspath{{c:/local/fig/}}                  %图片文件夹
...
\biblilographystyle{c:/local/IEEEtran}          %使用参考文献格式,IEEEtran.bst
\bibliography{c:/local/bib}                     %插入参考文献列表,bib.bib
\end{verbatim}

这样管理参考文献和图片,可以避免硬盘中存在多个文件版本,也方便写毕业论文使用。

注:在本模板的根目录下,存有一个bib.bib文件。



\section{图表}
\subsection{图片的插入方法}
单张图片独自占一行的插入形式如图~\ref{fig:dect_12}~所示。
\begin{figure}[htbp]
\centering
\includegraphics[width=0.4\textwidth]{ect_12.mps}
\caption{12电极ECT传感器}\label{fig:dect_12}
\vspace{\baselineskip}
\end{figure}



若需要将~2~张及以上的图片并排插入到一行中,则需要采用\verb|minipage|环境,如图~\ref{fig:ect}~和图~\ref{fig:dect}~所示。
\begin{figure}[htbp]
\centering
\begin{minipage}{0.4\textwidth}
\centering
\includegraphics[width=\textwidth]{ect_12.mps}
\caption{非差分ECT}\label{fig:ect}
\end{minipage}
\hspace{0.1\textwidth}
\begin{minipage}{0.4\textwidth}
\centering
\includegraphics[width=\textwidth]{dect_12.mps}
\caption{差分ECT}\label{fig:dect}
\end{minipage}
\vspace{\baselineskip}
\end{figure}


\subsection{具有子图的图片插入方法}

图中若含有子图时,需要调用~subfigure~宏包, 如图~\ref{fig:subfig}~所示。
\begin{figure}[htbp]
  \centering
  \subfigure[非差分ECT]{\label{fig:subfig:datadim}
                \includegraphics[width=0.4\textwidth]{ect_12.mps}}
  \subfigure[差分ECT]{\label{fig:subfig:datasize}
                \includegraphics[width=0.4\textwidth]{dect_12.mps}}
  \caption{ECT传感器结构}\label{fig:subfig}
\vspace{\baselineskip}
\end{figure}

若想获得插图方法的更多信息,参见网络上的~\href{ftp://ftp.tex.ac.uk/tex-archive/info/epslatex.pdf}{Using Imported Graphics in \LaTeX and pdf\LaTeX}~文档。



\section{表格}

表格应具有三线表格式,因此需要调用~booktabs~宏包,其标准格式如表~\ref{tab:ch1:tab1}~所示。
\begin{table}[htbp]
\caption{三线制表格}\label{tab:ch1:tab1}
\vspace{0.5em}\centering\wuhao
\begin{tabular}{ccccc}
\toprule[1.5pt]
$D$(in) & $P_u$(lbs) & $u_u$(in) & $\beta$ & $G_f$(psi.in)\\
\midrule[1pt]
 5 & 269.8 & 0.000674 & 1.79 & 0.04089\\
10 & 421.0 & 0.001035 & 3.59 & 0.04089\\
20 & 640.2 & 0.001565 & 7.18 & 0.04089\\
 5 & 269.8 & 0.000674 & 1.79 & 0.04089\\
10 & 421.0 & 0.001035 & 3.59 & 0.04089\\
20 & 640.2 & 0.001565 & 7.18 & 0.04089\\
 5 & 269.8 & 0.000674 & 1.79 & 0.04089\\
\bottomrule[1.5pt]
\end{tabular}
\vspace{\baselineskip}
\end{table}

