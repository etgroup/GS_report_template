\section{Method}


\subsection{Cylindrical capacitor}
The capacitance of a cylindrical capacitor is described by:

\begin{eqnarray}
C_{cc} = \frac{2\pi h \varepsilon_0}{\ln(b/a)}
\label{eq:cylindrical_capacitor}
\end{eqnarray}
where $\varepsilon_0$ is the dielectric permittivity of air, $h$ is the axial length of active screen, $a$ and $b$ are the radius of measuring electrodes and active screen, respectively.



Therefore, it can be assumed that

\begin{eqnarray}
C_{d1}&=&C_{d2}=\cdots=C_{di}\\
C_{cc}&=&\sum_{i=1}^{n_E} C_{di}
\label{eq:cd}
\end{eqnarray}
where $C_{di}$ is the capacitance between the $i^{th}$ measuring electrode  and the active screen.

In consideration of the gaps between measuring electrodes and radial screens and the edge effect caused by finite $h$, $C_{di}$  can be estimated by:

\begin{eqnarray}
C_{di} = k_c\frac{C_{cc}}{n_E}=\frac{2k\pi h \varepsilon_0}{\ln(b/a)n_E}
\label{eq:cna}
\end{eqnarray}
where $k_c$ is a correction factor related to sensor geometry, $n_E$ is the number of measuring electrodes.

\subsection{Circuit output}


Therefore, the output voltage can be described by:

\begin{eqnarray}
V_o(t) = -\frac{j\omega R_f(C_m - kC_{d2})}{j\omega C_f R_f +1}V_i(t)
\label{eq:voa}
\end{eqnarray}
where $\omega$ is the applied angular frequency, $R_f$ and $C_f$ are the feedback resistor and capacitor.

By introducing $k_0 = {j\omega R_f}/(j\omega C_f R_f +1)$, Eq(\ref{eq:voa}) can be simplified into:

\begin{eqnarray}
V_o(t) =-k_0(C_m - kC_{d2})V_i(t)
\label{eq:vob}
\end{eqnarray}


\subsection{Noise analysis}


In consideration that
\begin{eqnarray}
|j\omega R_{s3}(C_m+C_d+C_{s3}+C_{s4})| &\ll&1,\nonumber\\
|j\omega R_{s4}(C_m+C_d+C_{s3}+C_{s4})| &\ll&1,\nonumber\\
|j\omega R_fC_f| &\gg&1\nonumber
\label{eq:approx_cond}
\end{eqnarray}
the equations for noise gains can be properly simplified.

The total RMS noise can be expressed by

\begin{eqnarray}
E_{t} = \sqrt{\sum_{i=1}^{q} E_{n}^2(i)}[V_{RMS}]
\label{eq:vrms_sum}
\end{eqnarray}
where $E_{n}(i)$ represents the $i^{th}$ RMS noise source, \textit{i.e.} the passive elements and the intrinsic noise sources in the op-amp.

Each noise source generates a noise density at the inverting or non-inverting input of the op-amp, which can be  calculated by:

\begin{eqnarray}
E_n = \sqrt{\int_{0}^{\omega_c}|A_n(\omega)|^2 e_{n}^{2}(\omega)d \omega}[V_{RMS}]
\label{eq:vrms}
\end{eqnarray}
where $A_n(\omega)$ is the noise gain, $\omega_c$ is the cut-off frequency of the following conditioning circuit.

\subsection{Sensitivity distribution}

In ECT, the sensitivity distribution of electrode pair $(i, j)$ $\mathbf{S}_{i,j}(k)$ is defined as:

\begin{eqnarray}
\mathbf{S}_{i,j}(k)=\mu (k)\frac{C_{i,j}(k)-C_{i,j}^l}{\Delta C_{i,j}\Delta \varepsilon}
\label{eq:sensitivity}
\end{eqnarray}
where
\begin{eqnarray}
\Delta C_{i,j} &=&C_{i,j}^h-C_{i,j}^l\\
\Delta \varepsilon &=&\varepsilon^h-\varepsilon^l
\label{eq:sensitivity2}
\end{eqnarray}
and where $C_{i,j}(k)$ is the capacitance when the $k^{th}$ element has the high dielectric constant $\varepsilon^h$ and all the other elements have low dielectric constant $\varepsilon^l$. And $C_{i,j}^h$ and $C_{i,j}^l$ are the capacitances when the pipe is filled with  high and low permittivity materials, respectively. $\Delta C_{i,j}$ is, therefore, the full scale capacitance range.
